\documentclass[10pt,twocolumn]{article}   	% use "amsart" instead of "article" for AMSLaTeX format
\usepackage{geometry}                		% See geometry.pdf to learn the layout options. There are lots.
\geometry{a4paper}                   		% ... or a4paper or a5paper or ... 
%\geometry{landscape}                		% Activate for rotated page geometry
%\usepackage[parfill]{parskip}    		% Activate to begin paragraphs with an empty line rather than an indent
\usepackage{graphicx}				% Use pdf, png, jpg, or eps§ with pdflatex; use eps in DVI mode
								% TeX will automatically convert eps --> pdf in pdflatex		
\usepackage{amssymb}

%SetFonts

%SetFonts


\title{Norm erosion as accidents in complex socio-technical systems}
\author{Daniel P. Scarnecchia}
\date{October 2017}							% Activate to display a given date or no date

\begin{document}
\maketitle
\section{Introduction}
Reflecting on the causes of the Great War, Winston Churchill once
attributed much of the course of human affairs to accident rather than
deliberation \cite{}\^{}Churchill Great War{]}. This is a result of the
complexity of human society, the absence of perfect information by all
actors, and perhaps trust. Recently, the role of modern technology
platforms' impact on conflict, elections, and human rights has come into
question, and it is becoming increasingly apparently that these
platforms are complex artifacts with social and technological components
that may interaction unforeseen ways. This paper attempts to draw from
accident theory as it relates to highly complex systems and applies it
to these socio-technical systems. It begins by reviewing a theory of
complex system failure and the critiques of this theory, and draws on
applications of these theories in systems-theoretical approaches to
understanding accidents in complex systems. It then seeks to provide
examples of how complex modern technology is creating ``accidents'' or
harms in it's interactions with human society which threaten to
undermine the global commons and suggests a course of action.

\section{Normal Accidents}\label{normal-accidents}

Normal Accident Theory (NAT), first articulated by sociologist Charles
Perrow following the 1979 Three Mile Island nuclear accident, discusses
a class of accidents which are ``normal'' or unavoidable in highly
complex systems.\cite{Perrow1999} In these systems, multiple components or
processes interact in complex ways. This \emph{interactive complexity}
may yield unexpected interactions, and as the complexity of a system
increases, so too does the likelihood of multiple, unrelated failures.
In the case of redundancy, these unrelated but interdependent systems
can be said to be \emph{loosely coupled}. This is in contrast to
\emph{tightly coupled} components, where causality is directly
related.\cite[p. 6-8]{Perrow1999} He argues that in highly complex systems, smaller
failures compound, and can cascade into critical incidents. These
trivial failures are compounded by failures in loosely coupled
systems---such as backup systems or warning lights---and human
interactions with these systems. The speed, complexity, and
improbability of these failures often leads to errors in the heuristic
of a problem constructed by the operator---and even well trained
operators acting rationally based upon available information will
compound or fail to mitigate the problem---if they realize they're
happening. The pessimism of Perrow's proscriptions---that some systems
are too complex and too high risk to be built---is driven in part by the
fact that additional safety features increase complexity and make
failure more likely.

\subsection{The limits of Normal Accident
Theory}\label{the-limits-of-normal-accident-theory}

While Normal Accident Theory is useful for understanding how complex
systems fail, it has limitations, including: - the assumption that all
accidents are the result of component failures, and - the assumption
that redundancy is the only form of safety. While Normal Accident Theory
grapples with complex systems by introducing the concept of non-linear
and INUS failures as the cause of accidents, it suffers from the
assumption that accidents require failures. However, accidents can also
occur when components operate as intended, but interact with other
components or processes.\cite[p.12]{Marais} This may be seen as a design failure,
but in sufficiently complex systems it may be impossible to predict all
possible interactions between components or subsystems.

The second limitation is Perrow's assumption that redundancy is the only
possible solution to risk mitigation. As Marais, et al. points out, this
is only one possible response---and frequently the costliest. Other
approaches include reducing unnecessary interactive complexity,
de-coupling or reducing tight coupling, and the reducing the potential
for human error through standardization.\cite[p.2]{Marais} They continue by
pointing out that the problem is one of opportunity cost between
functionality and safety, as measured in risk. This is a
\emph{trans-scientific question}---that is to say, a problem that be
informed by science and engineering, but ultimately asks questions of
the domains of politics and ethics.\cite[p.3]{Marais}

\section{Human Systems}\label{human-systems}

This is true of organizations as well as engineered systems. Accidents
can be produced by both organizational failures, and dysfunction in
interactions between organizations.\cite[p.12]{Marais} It is through this lens
that researchers must interrogate the role of technology in society in
the 21st century, and begin to interrogate the unintended consequences
of certain technologies on governance and norms. From there we can begin
a conversation about what sorts of constraints and governance can answer
the challenges of these technologies in society.

In the accident literature, the interplay of technical systems and human
organizations and management structures is referred to as a
socio-technical system. Social systems are complex, and
include structure, culture, interaction dynamics, and individual factors
in understanding how they function. Further complexity is added when
factoring in how these complexities interact with highly complex
technical systems, and the management literature devotes much thought to
this. Systems theory emerged from a recognition that complex systems
yield highly complex, non-linear, and indirectly causal relationships
that can lead to accidents.\cite[p.13]{Marais}

In the 21st century, the reach and capacity of the internet has grown
quickly to cover much of the globe, and in so-called developed
countries, is now the dominant mode of communication. One of the more recent offshoot developments is social
networking platforms---over the last decade these platforms have grown
from novelties to behemoths with vast market capitalization and a user
base measured in billions of people. They may be changing human cognition in ways we don't entirely
understand, as well as changing the breadth and depth of human
relationships.

It is also increasingly possible that they may be fundamentally altering
human society. Perhaps now famously, Langdon Winner asked ``Do artifacts
have politics?'' He argues that this may be the case in two ways. In the
first, a specific technology may provide the means by which political
actors may capture institutional power. In the second, technologies may
exhibit properties which rearrange social and political
settlements.\cite{Winner1980} Socially networked communications systems
may exhibit aspects of both---and properties which creative positive
effects at the individual level may have pernicious effects on the
macro-level.

These platforms can be thought of as complex systems or systems of
systems, and demonstrate emergent behaviors---that is, behaviors that a
more than an aggregation of their components or subsystems.\cite[p.695.]{Haglich2010}
These behaviors, emerging from the interaction of subnetworks can yield
unforeseen or difficult to predict behaviors.\cite[p. p.693]{Haglich2010} Like other
highly complex systems, these may cause dysfunction or failures which
cascade into broader systemic failures or disruptions. Unlike most
engineered systems, this Global Networked Public is a complex
socio-technical system which doesn't just interact with individual
organizations, but with a large aggregate of humanity. Thus, the
implications may be global in scale.

\section{Ebola}\label{ebola}

A recent paper out of the MIT Center for Civic Media found that during
the 2014 Ebola crisis in West Africa, Twitter was the fourth most linked
media source of information. Unlike the CDC and WHO---which were the
first and second sources, respectively---the global discourse and
engagement on Twitter was focused on the risk posed by the twenty-five
or so U.S. domestic infections rather than those in West Africa. The
authors theorize that the global networked public is not simply an
audience, but a major actor in the spread of information and
misinformation. They go on to posit that this, in interaction with
traditional media incentives, may have affected policy and resource
responses around the response, and influenced the decision to violate
International Health Regulations.\cite[p. 51-52]{Roberts2017}

Other examples---to be written: 1. Starbird---Cyber/Information
War---Syria 2. Facebook: US Election, hate speech/targeted
advertisements, and the failure of algorithms---artificial evil 3.
Burma?

\section{A research agenda for the 21st
century}\label{a-research-agenda-for-the-21st-century}

None of these examples represent concrete evidence, but they speak to
the complexity of the interactions between a new global technology and
human society. This interaction exhibits the potential to undermine
protective norms, reconfigure the global commons in unpredictable ways,
and ultimately cause harm to people.

Marais et al, point to another emergent property in complex
systems---safety. Safety in systems is governed by constraints---that
is:

\begin{quote}
``\cite{}l{]}imitations on behavior which prevent unsafe behavior or unsafe
interactions among system components. For example, in a wartime
situation, a safety constraint on aircraft operations in order to
prevent friendly fire accidents is that the pilots must always be able
to identify the nationality of the other aircraft or ground troops in
the area around them. A large number of components of the military
``system'' each play a role in ensuring that this constraint is
satisfied.''\cite[p.13-14]{Marais}
\end{quote}

The turn towards safety in combat aircraft as an illustration is useful,
as the means to identify friend or foe parallels the principle of
distinction in international law, which mandates clear markings on
military materiel and personnel so as to distinguish between them and
civilians. Through this lens, International Law may
be seen as a set of constraints governing a highly complex system,
including the relatively chaotic one which constitutes warfare.
Nevertheless, human society is dynamic, and any set of constraints must
be able to adapt to changing processes and contexts. Turning again to
Marais, et al., who state that understanding and preventing accidents
requires: - identifying the necessary constraints to prevent accidents;
- designing systems to enforce constraints, and anticipating how the
constraints may be violated; and - determining how changes over time
potentially increase risks, and what constitutes unacceptable
risk.\cite[p. 13]{Marais}

Put in other terms, this might represent the later half of a research
agenda into how the potential effects of these new socio-technological
platforms on society might be managed. In the human context, accidents
might best be thought of as harms and constraints might best be thought
of as law. Harms at their most basic level can and must be understood as
violation of basic guaranteed human rights and dignity. These harms
should be seen as emerging from complex interactions between parts or
subsystems of socio-technical systems. They will continue to grow in
complexity, as billions of more people come online globally and as AI
and algorithms augment decision making.

Thus any research agenda must begin with the recognition that rights
apply in these new socio-technical systems, and it must grapple with how
these systems violate rights in a direct and indirect manner. Only then
can we understand how existing laws apply, where they fall short, and
what new constraints may be necessary to mitigate anticipatable and
anticipatable harms. Research into the technical aspects of these
systems can only yield so much insight. In order to attend to the human
and societal contexts in which these technical systems exist, research
much draw from political theory and economy, law, sociology, and
information behavior.\cite[p.12]{Marais}, \cite[p. 77]{Crawford2016}, \cite[p. 135]{Winner1980}

The agreements and that have governed the commons over the last eight
decades have seen the greatest measurable increase in human wellbeing
and dignity since the dawn of history, human rights and
liberal-democratic values have been enshrined in global governance, and
technology has given individuals new modes of free expression and
communication. But paradoxically, the fruits of socio-technological
progress now threaten that settlement---and the threat arises from
systems which are so complex, tightly coupled to human behavior, and
opaque that by the time it appears urgent to act, the system may have
already reached the point where critical failure is inevitable. Absent
the political will to understand and mitigate these harms is the risk of
consigning the global commons to instability, political capture, and
authoritarianism.

\bibliographystyle{acm}
\bibliography{references}

\end{document}  