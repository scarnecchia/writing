% Options for packages loaded elsewhere
\PassOptionsToPackage{unicode}{hyperref}
\PassOptionsToPackage{hyphens}{url}
%
\documentclass[
]{article}
\usepackage{lmodern}
\usepackage{amssymb,amsmath}
\usepackage{ifxetex,ifluatex}
\ifnum 0\ifxetex 1\fi\ifluatex 1\fi=0 % if pdftex
  \usepackage[T1]{fontenc}
  \usepackage[utf8]{inputenc}
  \usepackage{textcomp} % provide euro and other symbols
\else % if luatex or xetex
  \usepackage{unicode-math}
  \defaultfontfeatures{Scale=MatchLowercase}
  \defaultfontfeatures[\rmfamily]{Ligatures=TeX,Scale=1}
\fi
% Use upquote if available, for straight quotes in verbatim environments
\IfFileExists{upquote.sty}{\usepackage{upquote}}{}
\IfFileExists{microtype.sty}{% use microtype if available
  \usepackage[]{microtype}
  \UseMicrotypeSet[protrusion]{basicmath} % disable protrusion for tt fonts
}{}
\makeatletter
\@ifundefined{KOMAClassName}{% if non-KOMA class
  \IfFileExists{parskip.sty}{%
    \usepackage{parskip}
  }{% else
    \setlength{\parindent}{0pt}
    \setlength{\parskip}{6pt plus 2pt minus 1pt}}
}{% if KOMA class
  \KOMAoptions{parskip=half}}
\makeatother
\usepackage{xcolor}
\IfFileExists{xurl.sty}{\usepackage{xurl}}{} % add URL line breaks if available
\IfFileExists{bookmark.sty}{\usepackage{bookmark}}{\usepackage{hyperref}}
\hypersetup{
  pdftitle={Networked Digital ICTs and the Humanitarian Arena: Challenges to Current Humanitarian Practice and Doctrine},
  pdfauthor={Daniel Scarnecchia, Harvard University; Nathaniel A. Raymond, Yale University; Danielle N. Poole, Harvard University},
  hidelinks,
  pdfcreator={LaTeX via pandoc}}
\urlstyle{same} % disable monospaced font for URLs
\setlength{\emergencystretch}{3em} % prevent overfull lines
\providecommand{\tightlist}{%
  \setlength{\itemsep}{0pt}\setlength{\parskip}{0pt}}
\setcounter{secnumdepth}{-\maxdimen} % remove section numbering
\newlength{\cslhangindent}
\setlength{\cslhangindent}{1.5em}
\newenvironment{cslreferences}%
  {\setlength{\parindent}{0pt}%
  \everypar{\setlength{\hangindent}{\cslhangindent}}\ignorespaces}%
  {\par}

\title{Networked Digital ICTs and the Humanitarian Arena: Challenges to
Current Humanitarian Practice and Doctrine}
\author{Daniel Scarnecchia, Harvard University \and Nathaniel A.
Raymond, Yale University \and Danielle N. Poole, Harvard University}
\date{}

\begin{document}
\maketitle
\begin{abstract}
Much of the literature on humanitarian practice is focused on the uses
of digital technology by responders. This literature does not
consistently examine how such technology is being used by other actorsin
the humanitarian arena and its potential subsequent effects on
humanitarian outcomes. This paper explores how networked digital ICTs
may be mediating information in ways that directly affect the offline
nature of humanitarian crises. We draw on evidence from migration
studies, political science, and social network research to demonstrate
that these technologies have the potential to shape the ways in which
the displaced seek aid, the nature of crises, and humanitarian
governance. From this we call on the sector to establish a research
agenda which seeks to comprehensively understand the range of
effectscreated by networked digital ICTs on the outcomes of aid.
\end{abstract}

\textbf{\emph{Draft. Please do not cite or circulate without
permission.}}

\textbf{Keywords:} \emph{information and communications technology,
humanitarian governance, global governance, humanitarian arena,
displaced populations}

Since 2002, when the amount of information and data stored digitally
surpassed that of analogue storage, the amount of information stored,
processed, and communicated by humanity has grown by several orders of
magnitude (Hilbert and Lopez 2011, 62--64; Hilbert, n.d., 3--9). At the
same time, the number of unique mobile phone subscriptions is expected
to rise to 5.7 billion by decade's end. Of these connections, industry
trade group GSMA expects 75\% to be 3G/4G mobile broadband by the end of
the decade (Intelligence 2017, 6). Digital information communication
technologies (digital ICTs) are now being utilized across the
humanitarian programming cycle for disaster preparedness, needs
assessment, aid delivery, and evaluation. Additionally, many
crisis-affected populations, including displaced populations, are now
perceived as tech-savvy, possessing mobile phones and relying on Wi-Fi
networks and cellular coverage (``Migrants with Mobiles: Phones Are Now
Indispensable for Refugees'' 2017). The initial excitement about the
possibilities of digital ICTs has been tempered by an increased
awareness of the risks these technologies may create for the
crisis-affected (Kuner et al. 2017, 14) and a growing acknowledgement of
the need for a better understanding of how the crises-affected use these
technologies (ELRHA 2017, 46; Vinck, Bennett, and Quintanilla 2018,
13).\footnote{This exercise notes that 38\% of humanitarian innovation
  and research projects are technology and communications focused, just
  11\% of research projects focused on technology versus 72\% of
  innovation projects. Further, only 49\% of humanitarian research is
  peer reviewed, and 28\% do not discuss or disclose methodology. This
  points to a severe crisis of evidence regarding humanitarian
  technology. For the purposes of this report, ELHRA defined research
  as: ``systematic investigations in humanitarian policy and practice''
  and innovation as ``an ``An iterative process that identifies, adjusts
  and diffuses ideas for improving humanitarian action.''} However,
further theory development and research needs to be conducted on how the
global shift towards digital connectively effects the nature of crises,
humanitarian response, and subsequently the crisis-affected.

Vinck et. al (2018, 19) point out that humanitarian organizations are
just beginning to understand how technology---particularly networked
digital ICTs---are being instrumentalized to undermine trust in our
society, and the particularly worrying impacts it has on humanitarian
organizations and affected peoples. The aim of this article is extend
this to present a broader understanding of digital technology's
affordances and potential impacts on the humanitarian sector, the
crisis-affected, and humanitarian outcomes. This paper does not seek to
add to the rich literature on the risks of humanitarian technology use
and innovation to affected populations, per se (Kuner et al. 2017;
Sandvik, Jacobsen, and McDonald 2017). Rather, it aims to discuss the
mediating effects of new technologies on expressions of human agency
among the crisis-affected and its potential to create new sites of
contention within global politics which may shift the ordering of
humanitarian governance and affect the outcomes of aid.

These broader implications effect how humanitarians understand
technology use by the crisis-affected and other parties in the
humanitarian arena, and subsequently how they design interventions that
can meet the needs of digitally connected crisis-affected populations,
protection strategies to mitigate risks to the crisis-affected, and
policy designed to ameliorate efforts by ill-intentioned actors
attempting to instrumentalize aid to their own ends. In addition, there
are implications for how humanitarian agencies engage with large
technology companies, and advocate on behalf of the interests of
humanitarianism and the crisis-affected (Vinck, Bennett, and Quintanilla
2018, 19--20).

First, we examine the changing role of information in crises and look at
the emerging evidence that digital ICTs are becoming an important source
of information for displaced populations. In the second section, we draw
on sociological work explaining the emergent nature of communities and
activities through networked digital ICTs. We then explore how states
and other actors contest the humanitarian arena, order humanitarian
governance, and affect aid outcomes. The third section then examines
three areas of concern to humanitarians for evidence that networked
digital ICTs are affecting aid outcomes: migration, political violence,
and global politics. Finally, we draw on these examples to suggest a
humanitarian research agenda which seeks to comprehensively understand
the range of effects created by networked digital ICTs on
humanitarianism.

\hypertarget{the-changing-role-of-information-in-humanitarian-crises-and-responses}{%
\subsection{The changing role of information in humanitarian crises and
responses}\label{the-changing-role-of-information-in-humanitarian-crises-and-responses}}

The traditional view of humanitarian information flows is one where NGOs
gather needs and information from beneficiaries and send resources
downstream to those in need (Keen 2008, 156; Vinck, Bennett, and
Quintanilla 2018, 44). While some interventions, such as Communicating
with Communities (CWC), are predicated upon the notion that information
is integral to the provision of aid (Committee 2017, 16), these downward
flows of information are constructed around the idea of the affected as
beneficiary and the NGO as a monopolistic provider of information.
However, the displaced person is not merely a beneficiary of
information, but an active agent at the centre of a complex information
environment in which they may be seeking, using and transmitting
information.\footnote{For example, consider the interviews cited in
  Hannides, Bailey, and Kaoukji (2016), 25: ``Many participants with
  mobile access explained that direct contact with other refugees who
  had already made the journey gave them access to a trusted network.
  They trusted these sources, before and during their own journeys, for
  advice on the best routes, smugglers' contact details, places to stay
  on the journey, GPS coordinates and how to avoid police to ensure they
  arrived at each location safely.''} This, in turn, has the potential
to shape the crisis space.

Deficits in information communicated by sources such as states and NGOs
may often leave displaced people with no alternative but to turn to
sources outside the organized response, such as smugglers, for
information (Hannides, Bailey, and Kaoukji 2016, 23). Wall et al.~(2017,
241--42) describe a state of \emph{information precarity} ``in which
{[}the affected population's{]} access to news as well as personal
information is insecure, unstable, and undependable, leading to
potential threats to their well-being.'' For example, in Foran and
Ianucci's (2017, 14) survey of migrants' information needs in Italy,
communication of information from official sources exists in an
information ecosystem in competition with unofficial sources. They found
failures in information provision were created by a broad interpretation
of what the law specifies needed to be supplied, a lack of translators
and cultural mediation, and lack of organization. Migrants were left
uninformed, and with their information needs unmet or informed by
inaccurate information, migrants engaged with alternative information
channels to fulfil their needs. For displaced persons residing in
countries without formally designated camp settlements, such as Lebanon,
reliance on informal sources of information can propagate misinformation
and rumours (``Isolated and Misinformed, Syrian Refugees Struggle''
2014). In other cases, such as in Uganda's Dadaab refugee camps, lack of
information was identified by camp residents as a barrier to accessing
basic needs such as food, shelter, and water (Internews 2011, 27).
Similarly, in Jordan's Za'atari camp refugees reported that a lack of
information hinders their access to food, medical care, and shelter
(Quintanilla 2012, 11).

These highlighted cases allude to how displaced populations, as well as
other populations affected by crises, are often in a state of extreme
information precarity, which affects decision-making and perceived and
real vulnerabilities. Emerging evidence suggests that at least some of
those affected by crises in this state appear to be turning to mobile
technology to meet their information needs: media reports suggest that
some displaced populations prioritize mobile technology over other
sources of aid (Brunwasser 2015) and that respondents perceive
connectivity as critical to their survival (Vernon, Deriche, and
Eisenhauer 2016, 14). Information precarity and unmet information needs
thus may be intrinsically tied to the human security of affected
populations.

What effect does mobile connectivity have on the overall human security
of specific populations compared to others? Emerging evidence of the
effects of connectivity on displaced populations offers some clues. A
BBC Media Action study of refugees in Greece and Germany showed that
those who maintained contact with families and had ``wide communication
networks'' were more resilient than those who did not. They felt that
greater access to information allowed them to better access aid and
allowed them to make decisions about their situation (Hannides, Bailey,
and Kaoukji 2016, 25). Oxford University's Humanitarian Innovation
Project shows that mobile connectivity may help grow and sustain refugee
economies (Betts et al. 2014, 33). In another case, preliminary results
from a study undertaken in a Syrian refugee camp in Greece show that
increased mobile phone use and access may be significantly associated
with a reduction in the probability of being moderately or severely
depressed (Poole 2017).

It is true too that while respondents to humanitarian surveys on
information sources frequently assert that mobile technology is widely
used, structural barriers to access may exist, including (but not
limited to) gender (Internews 2011), whether the population was urban or
rural (Vernon, Deriche, and Eisenhauer 2016, 12), and lack of
infrastructure and service (Schmitt et al. 2016). As such, without
further research into the barriers to and determinants of access, the
connectivity of affected and displaced populations cannot be
generalized, nor can the outcomes of aid be sufficiently predicted.

Further development of theory, evidence, and a professional
understanding of the effects of information precarity on affected
populations may be required to design and implement necessary
interventions meeting the needs of digitally networked affected
populations, particularly in refugee or displacement scenarios. Further,
aid outcomes may be shaped by the unintended effects of the affected
populations use of these technologies, as they may create new
vulnerability by revealing data about themselves (Vinck, Bennett, and
Quintanilla 2018). For these interventions to be effective and ethical,
further research is needed for humanitarians to be sufficiently
responsive to displaced populations that are digitally literate,
mobile-connected, and using technology to coordinate and survive.

\hypertarget{networked-digital-icts-and-the-humanitarian-arena}{%
\subsection{Networked Digital ICTs and the Humanitarian
Arena}\label{networked-digital-icts-and-the-humanitarian-arena}}

Before attempting to understand the barriers to access and information
needs, it is necessary to interrogate the networked nature of modern
digital ICTs---such as computers, mobile and smartphones---in the
context of human society. What we term in this paper as networked
digital ICTs are specifically those digital ICTs and platforms which
connect users in both a one-to-one and one-to-many fashion---e.g.,
social media platforms such as Facebook and messaging platforms such as
WhatsApp. These have the potential to change how displaced people
communicate and seek information, alter crises themselves, and affect
the global discourse around crises and humanitarian operations in ways
which affect the outcomes of aid.

\hypertarget{networked-and-algorithmic-publics}{%
\subsubsection{Networked and Algorithmic
Publics}\label{networked-and-algorithmic-publics}}

Unlike previous forms of information dissemination, such as the radio,
networked digital ICTs have fundamentally different properties affecting
information spread (Doerr, Fouz, and Friedrich 2012, 70). This is
because networks exhibit emergent behaviour (Haglich, Rouff, and Pullum
2010, 694), there is trust inherent in peer-to-peer sharing, and weak
tie networks have a propensity to introduce novel information into peer
groups (Granovetter 1973, 1360). The process by which this occurs is
fairly well understood. Social network theory identifies different types
of ties between individuals (Haythornthwaite 2002, 389). Strong ties are
close relationships, such as those between close friends or family
members. Weak ties are acquaintances, or distant relationships more
likely to be activated for a specific purpose rather than an emotional
bond. Latent ties are relationships that exist, in a mathematical sense,
but have not yet been activated, and require a social stimulus to do so.
Latent ties have the potential to become weak ties. Weak ties have been
demonstrated to be more important to people seeking information, as they
bridge social groups and become a primary source of non-redundant or
novel information (Granovetter 1973, 1360).

Connected populations of all sorts are actors engaging in communications
mediated by network technology. boyd (2011, 39) suggests that
``networked publics'' emerge from these networks and defines them as
``publics that are restructured by networked technologies. As such, they
are simultaneously (1) the space constructed through networked
technologies and (2) the imagined collective that emerges as a result of
the intersection of people, technology, and practice.''

Networked publics are made distinct by the ways technology shapes them
and creates new possibilities for engagement, interaction, and
information flows (boyd 2011, 39). Novel attributes of these
technologies alter the manner in which people engage in social life,
self-expression, and communication (boyd 2011, 48). They also introduce
new dynamics that alter the social context of participants regarding
privacy, place, and role (boyd 2011, 46).

The foundational vision of many of the global platforms which facilitate
networked publics is one in which access to more and higher quality
information by individuals will yield more informed decision-making by
individuals (Askonas 2019, 9). The resulting platforms were built on
engineering choices designed to foster interaction and engagement, and
make legible the social connections between users and ideas. These
architectural choices have not been without their downsides. The
emergence of algorithms as mediators of content within these platforms
have created effects which keep users engaged but subvert their
foundational visions. Algorithms are, of course, designed to many
ends---but many of those in the context of social media are to keep the
user engaged, largely by exaggerating confirmation bias and feeding the
user more of what they want to hear (Askonas 2019, 10). For example,
platforms providing users interested in politicized or controversial
content increasingly with extremist viewpoints which subsequently leads
viewers to trust more extremist political positions (Lewis 2018,
36--37).

The vision of connecting disparate groups has been subverted by this
phenomenon. The foundational promise of this technology was to connect
users across interest groups. In reality, users have been more tightly
connected to people sharing similar biases, affinities, and interests
(Askonas 2019, 10).These ``calculated publics'' are shaped by the
information that algorithms determine are worth sharing (Crawford 2016,
77). Subsequently, algorithmic mediation of content may create a
politics of their own.

\hypertarget{humanitarian-governance-and-the-humanitarian-arena}{%
\subsubsection{Humanitarian Governance and the Humanitarian
Arena}\label{humanitarian-governance-and-the-humanitarian-arena}}

In order to assess the effects of these new politics of the networked
and algorithmic public on humanitarianism and displaced populations, it
is first necessary to situate humanitarianism within global politics.
Barnett (2013, 381) provides a critical definition of humanitarian
governance as a ``project to `secure the welfare of the population, the
improvement of its condition, the increase of its wealth, longevity,
health,' and the betterment of its general well-being.'' He argues that
this definition allows for us to focus on the consequences of aid, and
understand this project as means of rule and power (Barnett 2013, 382).
Understanding humanitarian governance as a means of rule affords
provides us with two advantages. The first is that in considering the
effects of aid, we can distinguish between a normative idea of
humanitarianism, and humanitarianism as it exists in the world (Dijkzeul
and Sandvik 2019, 3). Hilhorst and Jenson (2010, 1117) contrast a
normative humanitarian space, constructed out norms and principles, and
a humanitarian arena in which a range of ``actors negotiate the outcomes
of aid (Hilhorst and Jansen 2010, 1120).''

The second is that in considering the question of the processes which
define what counts as humanitarian situation requiring outside
intervention, he allows for us to consider that the power of
humanitarian governance is contestable---viz.~subject to global power
politics (Barnett 2013, 382; Goddard and Nexon 2017, 10). Dijkzeul and
Sandvik (2019, 12) site humanitarian governance within the humanitarian
arena, and argue that many actors engaged in humanitarian action see
humanitarian activities as a good to be captured or a threat to be
countered. Hugo Slim (2003) puts it plainly:

\begin{quote}
``Humanitarianism is always politicized somehow. It is a political
project in a political world. Its mission is a political one---to
restrain and ameliorate the use of organised violence in human relations
and to engage with power in order to do so. Powers that are either
sympathetic or unsympathetic to humanitarian action in war always have
an interest in shaping it their way.\footnote{See also: Collinson and
  Elhawary (2012), p.3}''
\end{quote}

Platforms designed to facilitate the emergence of the networked public
and algorithmic mediate content can make legible and manipulatable the
connections between users and between users and ideas. These are in
effect, tools of surveillance and behaviour modification (Askonas 2019,
9). In the hands of ill-intended actors, these tools are being used to
monitor and control descent and they can provide leaders, states, and
political entrepreneurs with the ability to manipulate public discourse
and mobilize their own supporters (Askonas 2019, 11--12; Dreier and
Martin 2010, 763).

Like the analogue world, cyberspace is a political space with complex
power relationships, where the unique effects of networks can create new
power dynamics between individual actors, groups, and states (Betz and
Stevens 2011, 10--11, 108--11). In this regard, cyberspace is not a
neutral intermediary that transports information between devices.
Instead, it is capable of transforming, distorting, and imbuing meaning
as it mediates information between devices and users.\footnote{For a
  discussion of the distinction between intermediaries and mediators,
  see Latour (2005), 38; For readers interested in current debates
  around how cyberspace functions as a mediator and its role in
  constructing reality, see Couldry and Hepp (2017).} We will further
explore the relationship between cyberspace and the contestation of the
humanitarian arena in the next section.

\hypertarget{humanitarianism-in-the-networked-age-evidence}{%
\subsection{Humanitarianism in the Networked Age:
Evidence}\label{humanitarianism-in-the-networked-age-evidence}}

An examination of the impact of networked technology on three related
realms of inquiry---migration, political violence, and global
politics---can inform an understanding of how these technologies may be
transforming humanitarian crises and affecting humanitarian governance.
First, we focus on migration and displaced populations. The use of
networked digital ICTs by displaced peoples has the potential to change
how they communicate, seek information, and aid, and alter crises
themselves, and affect the global discourse around crises and
humanitarian operations. It also has the potential to upend the
traditional humanitarian hierarchy of provider and affected population
(Barnett 2013, 389), and it provides a means for as affected
populations, including the displaced, to become participants in the
humanitarian arena as they advocate for their own role in humanitarian
action (Vinck, Bennett, and Quintanilla 2018, 14 \& 64).

The section on political violence examines how networked publics have
the potential to create humanitarian crises. It takes considers the role
of networked digital ICTs in creating new politics within Myanmar which
may have affected the timing and severity of the Rakhine crisis.

Third, the section on global politics provides two cases as evidence of
the potential effects of networked digital ICTs and their emergent
publics on the humanitarian arena and how they may contribute to
humanitarian outcomes. The first case examines the role of
disinformation spread via these networked technologies in contesting the
humanitarian arena in Syria and its effect on humanitarian outcomes. The
second examines the role of the networked public in effecting the
positions and desired outcomes of states as actors in the humanitarian
arena during the 2014 Ebola response.

Migrants can be displaced populations, but not all displaced populations
are migrants. At the time of writing, much of the scholarly literature
examining the effects of the use of digital ICTs by migrant populations
are focused on migrants as a whole---particularly economic migrants. For
the purposes of this paper we ignore the distinction as a starting point
by which we can present evidence that digital ICTs and networked digital
ICTs affect population movements and decision-making. However, the
distinction between migrants as a whole and displaced populations is
salient: Displaced populations are often particularly vulnerable and are
governed by specific legal regimes.

The link between displaced populations and the effects of networked
digital ICTs on political violence and global politics is perhaps less
intuitive, but they are connected in operational reality and as loci for
contests over the consequences of humanitarian aid (Dijkzeul and Sandvik
2019, 12). As an operational reality, political violence is often a
source of displacement, and there is a nexus between the vulnerability
of displaced populations and political violence (``OHCHR Questions and
Answers About Idps,'' n.d.). Once a population is displaced, their
legitimacy within a secondary country or status as affected can be
contested by similar means. States and non-state actors pursue their
interests in the international political space, which may not always in
line with the best interests and rights of displaced populations or
humanitarian values. This is made manifest as political settlements to
resolve displacement and end conflicts may be blocked, the question of
whether it is safe for displaced populations to return to their country
of origin becomes politicized, or states which may seek to build support
for various interpretation of international law undermine the rights of
the displaced and crisis-affected.

\hypertarget{the-impact-of-digital-icts-on-migration-and-displaced-populations}{%
\subsubsection{The Impact of Digital ICTs on Migration and Displaced
Populations}\label{the-impact-of-digital-icts-on-migration-and-displaced-populations}}

Migrant networks as a whole and their use of networked digital ICTs have
been studied extensively in both the scholarly and grey literature
(Schapendonk and Moppes 2007). Even before the advent of digital
technologies, migrants have always communicated amongst themselves and
back home, and these social networks function as a source of information
and social capital (Dekker and Engbersen 2014). Increasingly, these
networks are thought to reduce risk, influence decision making, and
mobilize financial resources and collective intelligence in ways that
shape movement patterns across environments and influence outcomes
(Tilly 1991, 84--85; Poot 1996, 65--66; Massey and García España 1987,
737; Wissink and Mazzucato 2017 ; Boyd 1989 ; Thulin and Vilhelmson
2014, 389). These networks tend to extend themselves, and migrants often
follow other migrants (Tilly 1991, 86). While these networks have always
existed, theory and evidence suggests that their digitization may be
affecting the nature of migration.

Even before the advent of mobile internet, access to digital ICTs has
been shown to play a role in the development of weak ties between
aspiring migrants and the diaspora, specifically for seeking information
about migration (Hiller and Franz 2004, 738). For some time, the
internet has been a source of information for migrants---irregular or
otherwise---in the form of chat rooms, websites, and Bulletin Board
Services (Hiller and Franz 2004, 738--39). For example, a 2007 report
commissioned for UNDP references websites that reportedly provide
detailed guidance to aid in journeys, and forums for communicating with
other potential migrants (Hamel 2009, 17). With the advent of social
networking sites, the process of developing and maintaining these ties
has been made easier and is shown to lower the threshold for immigration
(Dekker and Engbersen 2014, 408).

Migrants use digital information sources as a tool for deciding where to
migrate, to reduce anxiety around decision making and as a mechanism for
finding support upon relocation (Ros et al. 2007 ; Thulin and Vilhelmson
2014, 395--96). These transformations facilitate the spread of
information and foster the creation of social capital by migrants
(Dekker and Engbersen 2014, 408; Hiller and Franz 2004, 748--49), which
may ultimately influence individuals' decisions to migrate in response
to information about the availability of jobs, housing, and settlement
assistance (Dekker and Engbersen 2014, 407).

In addition to information about the migration process, networked
digital ICTs are used by migrants in other ways. Refugees and migrants
have been documented using mobile phones to share information,
coordinate among each other upon arrival and monitor for perceived
threats (Harney 2013, 548). When Hannides et al.~(Hannides, Bailey, and
Kaoukji 2016, 25) asked refugees who have mobiles why they feel more
resilient and less vulnerable, they found:

\begin{quote}
``Many participants with mobile access explained that direct contact
with other refugees who had already made the journey gave them access to
a trusted network. They trusted these sources before and during their
own journeys, for advice on the best routes, smugglers' contact details,
places to stay on the journey, GPS coordinates and how to avoid police
to ensure they arrived at each location safely.''
\end{quote}

At the same time, some migrants view mobile technology as a potential
source of risk: an avenue for tracking and monitoring by authorities or
non-state armed groups and of abuse and extortion by criminal elements
(Newell, Gomez, and Guajardo 2016, 184--85).

Recent work applying quantitative social network analysis to migrant
networks finds that migrants with transnational networks receive more
information and financial resources than migrants with smaller local
networks (Bilecen and Cardona 2017, 9). Networked digital ICTs may play
a role in transforming migrant experiences into far more transnational
experiences, as the immediacy and density of their transnational
communications increase (Nedelcu 2012, 1346--52). Thus, mobile
connectivity may be strengthening latent and weak ties in the social
networks of migrants in a manner that influences their decision-making
and creates new benefits and risks.

\hypertarget{the-impact-of-digital-icts-on-manifestations-of-political-violence}{%
\subsubsection{The Impact of Digital ICTs on Manifestations of Political
Violence}\label{the-impact-of-digital-icts-on-manifestations-of-political-violence}}

Political violence and conflict are a major source of displacement, as
well as sources of increased vulnerability for the displaced (Hannides,
Bailey, and Kaoukji 2016, 25). While the literature on migration
demonstrates how individual behaviour may change due to the influence of
networked digital ICTs, as an aggregate, individuals constitute and
exist in larger publics, such as states. It follows then that networked
digital ICTs may affect society, including the outbreak of uprisings and
conflicts, and factor into the causes of displacement and humanitarian
crises.

In the aftermath of the Arab Spring and response to the perceived role
of mobile communications and social media in the genesis of those
protests, the political science literature has devoted discussion to the
role of ICTs and mobile communications coverage and the onset of
political violence and protest (Dafoe and Lyall 2015). This has produced
findings---albeit contradictory and limited---covering the relationship
between increased cellular network density and the onset of violence
(Pierskalla and Hollenbach 2013 ; Shapiro and Weidmann 2015 ; Bailard
2015) and has raised debate about whether connectivity has a
prophylactic effect---helping civilians avoid or prevent violence---or a
harmful effect, making violence more likely and more organized.

The contradictory nature of this evidence may exist in part because the
proxy for connectivity used in these studies is mobile communications
network coverage. In a 2G world, this proxy may have been sufficient for
making determinations about the impact of mobile ICTs on conflict, but
the expansion of networked digital ICTs and 3G technology may tell a
different story. Myanmar provides a telling example: A study by Bergren
and Bailard (2017) find no significant relationship between mobile phone
reception and the rate of violence between ethnic groups and the
government in Myanmar. However, the paper uses mobile communications
coverage as a proxy for mobile phone access (Bergren and Bailard 2017,
899), and Bergren and Bailard believe it is unclear how well their paper
relates to violence against the Rohingya (Bailard 2017).

Indeed, the UN posits a link between the rapid growth of connectivity in
Myanmar and the government's ethnic cleansing campaign against the
Rohingya in that country's Rakhine state (Council 2018, 16--18).
Evidence has emerged in the media of a coordinated effort by the Myanmar
military to deliberately use Facebook to incite violence and turn
society against the Rohingya (Mozur 2018).

Myanmar has leapfrogged from nearly zero connectivity to widespread
access to the internet. In 2011, fewer than .02\% of the country of 50
million was connected to the internet---by November 2016 that number had
reportedly soared to 35 million (Frenkel 2016). Facebook, through its
Free Basics program---in partnership with state telecom provider
MTN---is reportedly integral in connecting many of these to the
internet---the platform now has 30 million users in Myanmar---many of
whom don't interact with the internet outside of Facebook (Roose 2017).
Facebook has been implicated in the spread of anti-Rohingya and
anti-Muslim sentiment, hate speech, and disinformation by political
figures, extremists and everyday people in Myanmar (Specia and Mozur
2017 ; Beech 2017).

\hypertarget{the-impact-of-digital-icts-on-global-politics}{%
\subsubsection{The Impact of Digital ICTs on Global
Politics}\label{the-impact-of-digital-icts-on-global-politics}}

If the potential exists for networked digital ICTs to disrupt politics
on a national level, so too does the potential for networked digital
ICTs to affect international politics in ways which may negatively
affect vulnerable populations, including refugees and displaced
populations. This section seeks to illustrate that the global networked
public, constituted of networked digital ICTs, is a repertoire for
contestation of the humanitarian arena. This contestation, and the
resulting range of possible outcomes, has ramifications for all
categories of crises-affected populations, including the displaced. This
may result, for example, in the undermining of rights and protection of
civilians, or, as may have happened in the case of the Global Compact
for Migration, the undermining of the rights of the displaced (Cerulus
and Schaart 2019).

The role of networks in influencing global politics is also well
established. Carpenter rightly points out that relational ties between
actors are politically salient, and when understood this way are known
to affect transnational trade, conflict, and enterprise (Carpenter 2014,
566--81). Civil society advocacy and the phenomenon of issue emergence
in areas of human rights and law are particularly well-studied through
this lens (Price 2003 ; Carpenter 2007). Networks of actors are adept at
strategically deploy information to influence target organizations and
actors (Keck and Sikkink 1998, 2). Further, information technology as a
means by which civil society and other non-state actors strengthen
networks, fostering information exchange, and engage in advocacy
activities across borders (Keck and Sikkink 1998, 19--21, 28--32).
Technology makes geographic distances become less relevant and the need
for certain traditional elements of power---particularly human and
economic capital---is significantly reduced (Betz and Stevens 2011, 102;
Kramer, Starr, and Wentz 2009, 41).

For humanitarians, what may be most important is that technology's
effects can be harnessed by political actors to create new political
settlements (Sandvik 2016, 21). Of course, this isn't a new phenomenon
per se, nor is it the first-time networks can be said to have affected
the global order.\footnote{e.g., Religious networks during the European
  wars of religion may have forged new identities across national
  borders and constrained the abilities of leaders to mobilize against
  crises and reconfigured power relations in Europe. That these networks
  were pre-digital should not discount the potential impact of networked
  digital ICTs to reconfiguring power in the modern world---recalling
  that the printing press played an important, albeit non-deterministic,
  role in the spread of new religious ideas and identities in Early
  Modern Europe. The change in the scale and reduction in cost by which
  information which could be distributed following the invention and
  diffusion of Gutenberg's movable type into society has few analogues
  save the invention of radio, television, and the internet. For more
  see: Nexon (n.d.), 33; and Barzun (2001).} The novel dimension to
modern global social networks, however, is that unlike preceding mediums
that facilitated the formation of networks, these platforms have user
bases which can exceed the populations of even the largest nations on
earth (Zuckerberg 2017 ; Welch 2017 ; Lanchester 2017). This becomes
more relevant in an era, as ICRC President Peter Maurer (2019) notes,
where two trends threaten to undermine the communality and consensus
which forms the basis of International Humanitarian Law: regional and
global power competition and the emergence of cyberspace as a
battlefield.

Syria provides a case of state and non-state actors working
collaboratively to influence public discourse and affect strategic
outcomes: Media reports and research points to a sustained influence
campaign targetting Syria Civil Defence, also known as the White Helmets
(Solon 2017). Wilson, et. al.~(n.d.) find that this campaign is often a
collaborative effort between state media, western anti-war activists,
and alternative medias to produce narratives supporting the strategic
objectives of specific warring parties to the conflict or to countering
those of western human rights actors and states. These narratives
contest civilian status of the White Helmets under humanitarian law by
labelling them tools of foreign influence, terrorists, and blaming them
for chemical weapons attacks. These efforts have been
successful---online discourse is dominated by content meant to
delegitimize them and justify their killing (Starbird, Arif, and Wilson
2019).

This example demonstrates three ways in which the global networked
public is being used to contest the humanitarian arena in Syria. The
first is in controlling the outcomes of aid: Humanitarian aid has been
instrumentalized in the Syrian conflict (Parker 2013 ; ``Report of the
Independent International Commission of Inquiry on the Syrian Arab
Republic'' 2018a, 4; ``Report of the Independent International
Commission of Inquiry on the Syrian Arab Republic'' 2018b, 9 \& 16) and
local actors---such as the White Helmets---have sought to provide aid in
areas international humanitarian organizations cannot access (Abdelwahid
2013). In controlling narratives about these local actors, warring
parties are able to continue targeting them, and further constrain aid
to contested territories. The second is the use of these narratives to
contest the legitimacy of who gets to be a humanitarian in Syria---in
painting local actors as terrorists or tools of a foreign power, warring
parties and their online collaborators seek to undermine claims to
neutrality and impartiality. Third, in seeking to undermine the status
of these actors as civilians under international law, they seek to
constrain normative humanitarianism in a way that controls
humanitarianism outcomes.

This sort of use of networked digital ICTs can be understood as part of
what Goddard et al., call a repertoire of social disruption, a means by
which states and other actors to create division within an opponent's
population, with the aim of deterring action against their interests by
inhibiting the ability of opponents to collectively mobilize (Goddard,
MacDonald, and Nexon 2019, 10). States participate and shape this
environment in multiple ways---from leveraging flagship state media
outlets, supporting and amplifying friendly journalists, to pushing out
a range of content on aligned but ostensibly independent websites
(Starbird, Arif, and Wilson 2019, 13). Thus the means by which these
narratives are spread are also salient: Narratives are packaged into
campaigns resemble astroturfing campaigns, repackaging content to appeal
across ideological and political divides (Starbird et al. 2018). This
games the media ecosystem by flooding search results with preferred
narratives, as well as ultimately undermining trust in available
information (Starbird, Arif, and Wilson 2019, 13). Put plainly, digital
networked ICTs provide a toolkit by which ill-intentioned actors can
create uncertainty in states or organizations which would otherwise be
invested in normative humanitarian approaches by using them ``to
identify and target fragmented populations, to feed them disrupting
information more directly, and isolate audiences from competing claims
(Goddard, MacDonald, and Nexon 2019, 10).''

The unique topological features of digital networks increase the speed
and distance at which information and misinformation can spread (Doerr,
Fouz, and Friedrich 2012, 70). Opportunistic actors may be able to
exploit this to convert misinformation into disinformation campaigns
(Sunstein and Vermeule 2009). These campaigns often attempt to challenge
or undermine dominant media narratives (Starbird 2017, 10). Indeed, the
use of networked digital ICT platforms by state and non-state
organizations to achieve strategic goals related to political,
geopolitical, or military outcomes by influencing public discourse is
now well documented (Weedon, Nuland, and Stamos 2017, 4).

It is also the case that the networked digital ICTs---through the
mediation of the global networked public---may affect humanitarian
action in ways that are not all together as deliberate as
instrumentalization by warring parties or states. Roberts et al.~note
that the widely discussed social media clusters oriented towards US
politics may have led US policymakers to act and impose quarantines---in
violation of International Health Regulations (IHR) (Roberts et al.
2017, 53--55). They argue the global networked public itself may have
produced emergent effects that shaped the discourse around Ebola and
ultimately affected policy outcomes, including the diversion of critical
resources and the violation of IHR (Roberts et al. 2017, 51--52). Their
research showed a disconnect between global health experts---who were
influential among the traditional media and the discourse among the
general public---whom global health experts failed to influence. Within
social media, public discussion was disproportionately oriented towards
the risk of US domestic cases. Thus, new politics may emerge organically
from network behaviour (Haglich, Rouff, and Pullum 2010), and intersect
with traditional media and polity in a manner which the Roberts et
al.~(2017) argue impose constraints on humanitarian response.

Humanitarianism exists in a world that is both political and densely
networked in the digital sense. These networks comprise a complex
system, capable of emergent behaviours which may exhibit a politics of
their own (Haglich, Rouff, and Pullum 2010, 695). Networked digital
ICTs---may create centres of institutional power that are wielded within
the political space governing humanitarian response (Sandvik 2016, 21),
and which may be manipulated by actors who are hostile to or seek to
co-opt influence the humanitarian arena. ICTs may also create emergent
loci of power with the ability to affect public discourses in such a way
that drives policymakers to act---even in violation of international
law. Vinck, et al.~(Vinck, Bennett, and Quintanilla 2018, 44) note that
humanitarians need to better understand a potentially contentious and
divided media ecosystem. Their focus is on the potential for the erosion
of trust between humanitarians and local populations. We argue that,
similarly, humanitarians need to better understand how the global
networked public and networked digital ICTs can undermine normative
humanitarian action and be used to manipulate the humanitarian system.

\hypertarget{a-research-agenda-for-humanitarianism-in-the-digital-age}{%
\subsection{A Research Agenda for Humanitarianism in the Digital
Age}\label{a-research-agenda-for-humanitarianism-in-the-digital-age}}

In the previous subsections, we have provided examples of the potential
effects of networked digital ICTs on the humanitarian arena, and how
these may in turn affect the nature of the crises and the outcomes of
aid. In order to acheive a more full understanding, the humanitarian
sector needs to pursue a comprehensive research agenda focused on the
range of effects networked digital ICTs are having on humanitarian
operations and outcomes. In the next two subsections, we suggest two
possible areas of research. The first seeks to understand the barriers
to ICT access and use among displaced populations, as well as defining
what constitutes information needs among the affected. The second seeks
to understand on how other actors---armed groups, NGOs, advocates,
political actors, and states---use networked digital ICTs to contest the
humanitarian arena, order humanitarian governance, and shape the
outcomes of aid. This agenda should not focus on how humanitarian actors
are using these technologies, as much research already exists in this
space.

\hypertarget{barriers-and-facilitators-to-information-technology-access-and-use}{%
\subsubsection{Barriers and facilitators to information technology
access and
use}\label{barriers-and-facilitators-to-information-technology-access-and-use}}

The emergence ofcrisis-effected networked publics may give affected
populations a new voice for advocacy, and digital ICT-mediated networks
may create new opportunities for the displaced to find aid, self-rescue
and communicate with humanitarians. It may also create new risks and
harms for these populations as they use networked digital ICTs to fulfil
their information needs (Vinck, Bennett, and Quintanilla 2018). However,
without a clear understanding of the determinants of access to these
networks---such as gender, age, socio-economic status, or information
preferences, information needs cannot be appropriately evaluated, and
humanitarians run the risk of not perceiving the needs of vulnerable
unconnected communities, even within connected populations.

Existing evidence on displaced populations' medium of choice for
obtaining information seems to indicate that these affected populations
have varying preferences regarding the type of communication tools that
they use. This is in part dependent on who is asked, and in part
possibly determined by the time and location. In some instances, radio
may be the preferred mode of information access (Internews 2011); in
others, it may be television and word of mouth (``Inter-Agency Rapid
Assessment Report: Understanding the Information and Communication Needs
Among Idps in Northern Iraq'' 2014, 15). Mobile telephones may be seen
as less trustworthy sources of information than others (Quintanilla 2012
; Internews 2011) or even as a vector for potential threats (Maitland
and Xu 2015 ; Newell, Gomez, and Guajardo 2016, 184--85). In Italy,
where migrants originate from Africa, the Middle East, and Central and
South Asia, the utilized sources of information vary widely and include
word of mouth, Facebook, messaging applications, mobile telephony, and
satellite TV. Social media access plays a major role in facilitating the
communications needs of many of the migrants surveyed by Internews in
Italy in 2016 (Foran and Iacucci 2017, 12). This same report finds that
smugglers dominate the information landscape in both analogue and
digital networked communication---that is, word of mouth and social
media communication---and that information access varies by country of
origin for a number of reasons (Foran and Iacucci 2017, 12--13).

Vernon et. al found parity between the global average and urban refugees
in terms of access to 3G networks, but access significantly lags the
global average among rural refugee populations, regarding both 3G access
and mobile access overall. It also finds that most refugees own a mobile
phone, and nearly 39\% have a phone capable of internet access (Vernon,
Deriche, and Eisenhauer 2016, 12). Nevertheless, divides in access to
mobile phones persist, and not enough information is available to
determine a conclusive reason for these disparities. In 2015, 89\% of
refugees at Za'atari Camp in Jordan reported owning a mobile phone
(Schmitt et al. 2016, 25); however, 52\% of refugees surveyed arriving
in Greece that same year said they did not have access to a phone (UNHCR
2015, 15). In their study of Za'atari, Maitland and Xu observed that the
relative wealth of Syrian refugees made them more likely to use mobile
technology to connect to the internet than refugees surveyed in Rwandan
camps they visited, which were limited to voice and text on mobile
platforms (Maitland and Xu 2015, 7). Maitland and Xu also find that
gender did not affect mobile phone ownership rates in Za'atari (Maitland
and Xu 2015, 8), however, information preference and needs surveys
carried out by Internews found that gender divides exist in both
Northern Iraq and Kenya's Dadaab camps (``Inter-Agency Rapid Assessment
Report: Understanding the Information and Communication Needs Among Idps
in Northern Iraq'' 2014 ; Internews 2011).

Within the context of information behaviours by displaced people,
attention should be paid to the behaviours around and barriers to access
related to networked digital. The literature has begun to show that a
number of social factors may play into digital exclusion among
crisis-affected populations. For example, Wall \emph{et al.}'s (Wall et
al. 2017) link between information precarity and the stability and
security of a person is a complex one, with many intervening factors and
a range of potential outcomes which need to be better understood. Wall
suggests a context-driven theory: the credibility of information about
accessing aid in the camp is viewed by the affected through a history of
scepticism of information from official channels due to a lifetime in an
authoritarian country where misinformation and propaganda were the
norms.

Given the potentially transformative effect of these networked digital
ICTs on humanitarian outcomes, it is critical to understand these
barriers and facilitators. Poole's research demonstrates a significant
link between mental health outcomes and phone access in affected
populations. However, a significant barrier to access was ownership,
which was in turn associated with gender. Returning to information
seeking behaviour models, this socio-cultural barrier mediates an
individual's ability to access technology, which in turn affects their
access to information and may ultimately impact their health (Poole
2017). Additional literature suggests that access to mobile phones
improves economic outcomes and resilience in crisis-affected populations
(Hannides, Bailey, and Kaoukji 2016 ; Betts et al. 2014). These studies
need to be supplemented by further scholarly research that explicitly
measures phone access against additional health outcomes, and other aid
metrics, such as food security.

Understanding the information landscape will give humanitarians evidence
for how to better design response. The available evidence already
indicates that information related needs may be going unmet. For
example, the strategy of using and sharing multiple SIM cards---a
practice Maitland noted was used by the displaced in Za'atari to avoid
perceived Syrian surveillance, avoid network congestion and manage
costs---makes it difficult for camp staff to push out information to
camp residents via SMS or telephone. Camp economies have been seen to
change in Za'atari, driven by this trade and the need for phone charging
stations and internet access. The physical geography of camps will
affect the ability of displaced populations reliant on mobile and smart
phones to access information. It is unclear is how this affects social
and economic stratification within camps (Maitland and Xu 2015).

An information need is inherently instrumental, in that it is a
recognized or perceived gap in knowledge or information thought
necessary to achieve a desired end state or goal (Green 1990). In turn,
information seeking is a conscious activity, designed to fill the gaps
required to achieve a goal or reduce uncertainty around such a pattern
or gap (Case and Given 2016, 89--91). To place information precarity in
context, and to address its effects, it is necessary to understand the
barriers to information access---social, physical, and
psychological---as well as the motivations, needs and uses of
information in the context of humanitarian crises, particularly
sustained crises. Understanding the factors that both frustrate and
satisfy displaced people's information needs is critical to improving
humanitarian response. Without evidence guiding the diagnosis of a
displaced populations information practices and eco-system, the
humanitarian responder is unable to claim that they are adequately
meeting needs in the 21st century.

\hypertarget{digital-networked-icts-and-contestation-in-the-humanitarian-arena}{%
\subsubsection{Digital Networked ICTs and Contestation in the
Humanitarian
Arena}\label{digital-networked-icts-and-contestation-in-the-humanitarian-arena}}

Langdon Winner (1980, 123) famously asked: ``Do artifacts have
politics?'' He argued that this might be the case in two ways. In the
first, a specific technology may provide the means by which political
actors may capture power---or in his language ``settle an issue.'' In
the second, technologies may exhibit properties which rearrange social
and political systems. Networked digital ICTs may display aspects of
both---and properties which create positive effects at the individual
level may have harmful effects on the macro-level. This may represent a
paradox of protection: While these technologies may give individual
displaced people new tools for wayfaring, decision making, and advocacy,
the aggregate effects of these same technologies may also erode
protective norms and produce political settlements that constrain
humanitarians' ability to respond.

Increasingly, the lines between offline and digital life are blurring
(Floridi 2014), and the interplay between society and technology may be
producing new politics and means of contesting the humanitarian arena
and ordering humanitarian governance. This paper's sections on political
violence and humanitarian governance show that networked digital ICTs
may exacerbate violence, inflame political tensions, or shift policy
choices which may in turn impact the human security status of affected
individuals and humanitarian operations---as demonstrated by Twitter's
purported impact on the Ebola respons (Roberts et al. 2017) and the role
of social media in shaping the humanitarian arena in Syria (Starbird et
al. 2018 ; Starbird, Arif, and Wilson 2019). This assemblage of the
human and digital has the potential to create additional possible
vectors for harm---the production of mutually supporting alternative
narratives based on misinformation or disinformation (Starbird 2017, 5)
and of politics which explicitly impact the human security status of the
affected.

The questions raised here are broader than humanitarianism. As a sector
whose normative \emph{raison d'etre} is the prevention of suffering and
alleviation of harm, humanitarianism stands in a position where it can
see the emergence of these effects on the most vulnerable of
populations. It is critical for humanitarians to examine not just
individual components of the system, but to foster testable theoretical
approaches to understanding how harmful effects emerge, order
humanitarian governance, and affect the outcomes of aid.

\hypertarget{conclusion}{%
\subsection{Conclusion}\label{conclusion}}

In the sections above, this paper has looked to literature outside the
humanitarian field to draw inferences about phenomena that may be
affecting humanitarian crises, including: how displaced populations may
be seeking information and the role that networked digital ICTs have in
meeting and changing their information needs, as well as the potential
barriers to access. We also discussed how access to networked digital
technology may affect the decision-making of the displaced, as well as
potentially affect their ability to self-advocate, thus affecting the
humanitarian arena and strategies required on the part of humanitarians
to achieve ideal aid outcomes.

We also discussed how these how networked digital ICTs are changing
crises on a larger scale. First, by raising unanswered questions about
their role in fostering conflict and subsequently generating or
exacerbating existing humanitarian crises. Second, by looking at how
they may alter the operational and political landscape in which crises
occur and be used as means for contesting the humanitarian arena and
ordering humanitarian governance.

This paper highlighted a series of gaps in the evidence and knowledge
confronting humanitarianism related to the use of networked digital
ICTs. To comprehend these changes, it is necessary to develop an
understanding of how networked digital ICTs are transforming the
humanitarian arena. Our understanding of the impact of networked digital
ICTs on the crisis-affected, including the displaced, the humanitarian
arena, and the resulting effects on humanitarian outcomes is limited. It
is incumbent upon the humanitarian sector to foster a research agenda
for defining the information needs of displaced populations,
understanding how they are using networked digital ICTs to fulfil them,
how access to these technologies may be changing their decision-making,
and how their use of these technologies, in hand with how other actors
in the humanitarian arena, including states, armed groups, and
humanitarians, is affecting humanitarian outcomes pertaining to the
needs and protection matrix of the displaced and indeed all
crisis-affected populations. Addressing and understanding these
questions may determine what humanitarianism means and does in the 21st
century.

\hypertarget{references}{%
\subsection*{References}\label{references}}
\addcontentsline{toc}{subsection}{References}

\hypertarget{refs}{}
\begin{cslreferences}
\leavevmode\hypertarget{ref-Abdelwahid2013You}{}%
Abdelwahid, Dalia. 2013. ``"You Got the Stuff?": Humanitarian Activist
Networks in Syria.'' \emph{ODI HPN}, no. 59 (November): 15--17.

\leavevmode\hypertarget{ref-Askonas2019How}{}%
Askonas, Jon. 2019. ``How Tech Utopia Fostered Tyranny.'' \emph{The New
Atlantis}, no. Winter: 11.

\leavevmode\hypertarget{ref-Bailard2015Ethnic}{}%
Bailard, Catie Snow. 2015. ``Ethnic Conflict Goes Mobile: Mobile
Technology's Effect on the Opportunities and Motivations for Violent
Collective Action.'' \emph{Journal of Peace Research} 52 (3, SI):
323--37. \url{https://doi.org/10/f7bv9x}.

\leavevmode\hypertarget{ref-Bailard2017So}{}%
---------. 2017. ``So, Not Clear How It Relates to Current Rohingya
Events, but Was Just Thinking Today This May Be a Good Follow-up
Study.'' \emph{Twitter}.
\url{https://twitter.com/catie_bailard/status/923365522829463553?s=20}.

\leavevmode\hypertarget{ref-Barnett2013Humanitarian}{}%
Barnett, Michael N. 2013. ``Humanitarian Governance.'' \emph{Annual
Review of Political Science} 16 (1): 379--98.
\url{https://doi.org/10.1146/annurev-polisci-012512-083711}.

\leavevmode\hypertarget{ref-Barzun2001From}{}%
Barzun, Jacques. 2001. \emph{From Dawn to Decadence: 500 Years of
Western Cultural Life, 1500 to the Present}. New York, N.Y: Perennial.

\leavevmode\hypertarget{ref-Beech2017Across}{}%
Beech, Hannah. 2017. ``Across Myanmar, Denial of Ethnic Cleansing and
Loathing of Rohingya.'' \emph{The New York Times}, October.
\url{https://www.nytimes.com/2017/10/24/world/asia/myanmar-rohingya-ethnic-cleansing.html}.

\leavevmode\hypertarget{ref-Bergren2017Information}{}%
Bergren, Anne, and Catie Snow Bailard. 2017. ``Information and
Communication Technology and Ethnic Conflict in Myanmar: Organizing for
Violence or Peace?'' \emph{SOCIAL SCIENCE QUARTERLY} 98 (3): 894--913.
\url{https://doi.org/10/gbwhb7}.

\leavevmode\hypertarget{ref-Betts2014Refugee}{}%
Betts, Alexander, Louise Bloom, Josiah Kaplan, and Naohiko Omata. 2014.
``Refugee Economies: Rethinking Popular Assumptions.'' Oxford:
Humanitarian Innovation Project, University of Oxford.
\url{http://www.rsc.ox.ac.uk/files/publications/other/refugee-economies-2014.pdf}.

\leavevmode\hypertarget{ref-Betz2011Cyberspace}{}%
Betz, David J., and Tim Stevens. 2011. \emph{Cyberspace and the State:
Toward a Strategy for Cyber-Power}. Adelphi (Series) (International
Institute for Strategic Studies). Abingdon: Routledge for the
International Institute for Strategic Studies.

\leavevmode\hypertarget{ref-Bilecen2017Do}{}%
Bilecen, Başak, and Andrés Cardona. 2017. ``Do Transnational Brokers
Always Win? A Multilevel Analysis of Social Support.'' \emph{Social
Networks}, March. \url{https://doi.org/10/gc8fjz}.

\leavevmode\hypertarget{ref-boyd2011Social}{}%
boyd, danah. 2011. ``Social Network Sites as Networked Publics:
Affordances, Dynamics, and Implications.'' In \emph{Networked Self:
Identity, Community, and Culture on Social Network Sites}, edited by
Zizi Papacharissi. New York, NY: Routledge.
\url{http://www.danah.org/papers/2010/SNSasNetworkedPublics}.

\leavevmode\hypertarget{ref-Boyd1989Family}{}%
Boyd, Monica. 1989. ``Family and Personal Networks in International
Migration: Recent Developments and New Agendas.'' \emph{International
Migration Review} 23 (3): 638--70. \url{https://doi.org/10/fhwdr3}.

\leavevmode\hypertarget{ref-Brunwasser201521st-Century}{}%
Brunwasser, Matthew. 2015. ``A 21st-Century Migrant's Essentials: Food,
Shelter, Smartphone.'' \emph{New York Times}, August.
\url{http://www.nytimes.com/2015/08/26/world/europe/a-21st-century-migrants-checklist-water-shelter-smartphone.html?_r=0}.

\leavevmode\hypertarget{ref-Carpenter2007Setting}{}%
Carpenter, R. Charli. 2007. ``Setting the Advocacy Agenda: Theorizing
Issue Emergence and Nonemergence in Transnational Advocacy Networks.''
\emph{International Studies Quarterly} 51 (1): 99--120.
\url{https://doi.org/10.1111/J.1468-2478.2007.00441.X}.

\leavevmode\hypertarget{ref-Carpenter2014Lost}{}%
---------. 2014. \emph{Lost Causes: Agenda Vetting in Global Issue
Networks and the Shaping of Human Security}. Cornell University Press.
\url{http://cornell.universitypressscholarship.com/view/10.7591/cornell/9780801448850.001.0001/upso-9780801448850}.

\leavevmode\hypertarget{ref-Case2016Looking}{}%
Case, Donald O., and Lisa M. Given. 2016. \emph{Looking for Information:
A Survey of Research on Information Seeking, Needs, and Behavior}. 4th
ed. Bingley, UK: Emerald Publishing Group.

\leavevmode\hypertarget{ref-Cerulus2019How}{}%
Cerulus, Laurens, and Eline Schaart. 2019. ``How the Un Migration Pact
Got Trolled.'' \emph{POLITICO}.
\url{https://www.politico.eu/article/united-nations-migration-pact-how-got-trolled/}.

\leavevmode\hypertarget{ref-Collinson2012Humanitarian}{}%
Collinson, Sarah, and Samir Elhawary. 2012. ``Humanitarian Space: A
Review of Trends and Issues.'' Overseas Development Institute
Humanitarian Policy Group.
\url{https://www.odi.org/sites/odi.org.uk/files/odi-assets/publications-opinion-files/7643.pdf}.

\leavevmode\hypertarget{ref-InternationalRescueCommittee2017Using}{}%
Committee, International Rescue. 2017. ``Using Ict to Facilitate Access
to Information and Accountability to Affected Populations in Urban
Areas.''

\leavevmode\hypertarget{ref-Couldry2017mediated}{}%
Couldry, Nick, and Andreas Hepp. 2017. \emph{The Mediated Construction
of Reality}. Cambridge, UK ; Malden, MA: Polity Press.

\leavevmode\hypertarget{ref-UnitedNationsHumanRightsCouncil2018Report}{}%
Council, United Nations Human Rights. 2018. ``Report of the Independent
International Fact-Finding Mission on Myanmar.'' United Nations Human
Rights Council.
\url{https://www.ohchr.org/Documents/HRBodies/HRCouncil/FFM-Myanmar/A_HRC_39_64.pdf}.

\leavevmode\hypertarget{ref-Crawford2016Can}{}%
Crawford, Kate. 2016. ``Can an Algorithm Be Agonistic? Ten Scenes from
Life in Calculated Publics.'' \emph{Science, Technology, \& Human
Values} 41 (1). \url{https://doi.org/10/gddv8j}.

\leavevmode\hypertarget{ref-Dafoe2015From}{}%
Dafoe, Allan, and Jason Lyall. 2015. ``From Cell Phones to Conflict?
Reflections on the Emerging Ict--Political Conflict Research Agenda.''
\emph{Journal of Peace Research} 52 (3): 401--13.
\url{https://doi.org/10/f7bt37}.

\leavevmode\hypertarget{ref-Dekker2014How}{}%
Dekker, Rianne, and Godfried Engbersen. 2014. ``How Social Media
Transform Migrant Networks and Facilitate Migration.'' \emph{Global
Networks} 14 (4): 401--18. \url{https://doi.org/10/f6jc62}.

\leavevmode\hypertarget{ref-Dijkzeul2019world}{}%
Dijkzeul, Dennis, and Kristin Bergtora Sandvik. 2019. ``A World in
Turmoil: Governing Risk, Establishing Order in Humanitarian Crises.''
\emph{Disasters} 43 (S2): S85--S108. \url{https://doi.org/10/gfv692}.

\leavevmode\hypertarget{ref-Doerr2012Why}{}%
Doerr, By Benjamin, Mahmoud Fouz, and Tobias Friedrich. 2012. ``Why
Rumors Spread so Quickly in Social Networks.'' \emph{Communications of
the ACM} 55 (6). \url{https://doi.org/10/gdqfhj}.

\leavevmode\hypertarget{ref-Dreier2010How}{}%
Dreier, Peter, and Christopher R. Martin. 2010. ``How Acorn Was Framed:
Political Controversy and Media Agenda Setting.'' \emph{Perspectives on
Politics} 8 (3): 761--92.
\url{https://doi.org/10.1017/s1537592710002069}.

\leavevmode\hypertarget{ref-ELRHA2017Global}{}%
ELRHA. 2017. ``Global Prioritisation Exercise for Research and
Innovation in the Humanitarian System: Phase One Mapping.'' Global
Prioritisation Exercise for Research and Innovation in the Humanitarian
System. Cardiff: ELRHA.
\url{http://www.elrha.org/wp-content/uploads/2017/03/Elrha-GPE-Phase-1-Final-Report_Nov-2017.pdf}.

\leavevmode\hypertarget{ref-Floridi20144th}{}%
Floridi, Luciano. 2014. \emph{The 4th Revolution: How the Infosphere Is
Reshaping Human Reality}. First edition. New York, NY: Oxford University
Press.

\leavevmode\hypertarget{ref-Foran2017Lost}{}%
Foran, Rose, and Anahi Ayala Iacucci. 2017. ``Lost in Translation: The
Misinformed Journey of Migrants Across Italy.'' Internews.
\url{http://internews.org/sites/default/files/Internews_Lost_In_Translation_Publication_2017-05-23.pdf}.

\leavevmode\hypertarget{ref-Frenkel2016This}{}%
Frenkel, Sheera. 2016. ``This Is What Happens When Millions of People
Suddenly Get the Internet.'' \emph{BuzzFeed}.
\url{https://www.buzzfeed.com/sheerafrenkel/fake-news-spreads-trump-around-the-world}.

\leavevmode\hypertarget{ref-Goddard2019Repertoires}{}%
Goddard, Stacie E, Paul K MacDonald, and Daniel H Nexon. 2019.
``Repertoires of Statecraft: Instruments and Logics of Power Politics.''
\emph{International Relations} 33 (2): 1--18.
\url{https://doi.org/10/gf7rc9}.

\leavevmode\hypertarget{ref-Goddard2017Dynamics}{}%
Goddard, Stacie E., and Daniel H. Nexon. 2017. ``The Dynamics of Global
Power Politics: A Framework for Analysis.'' \emph{Journal of Global
Security Studies} 1 (1): 4--18.
\url{https://doi.org/10.1093/JOGSS/OGV007}.

\leavevmode\hypertarget{ref-Granovetter1973Strength}{}%
Granovetter, Mark. 1973. ``The Strength of Weak Ties.'' \emph{American
Journal of Sociology} 78 (6).
\url{https://sociology.stanford.edu/sites/default/files/publications/the_strength_of_weak_ties_and_exch_w-gans.pdf}.

\leavevmode\hypertarget{ref-Green1990What}{}%
Green, Andrew. 1990. ``What Do We Mean by User Needs?'' \emph{British
Journal of Academic Librarianship} 5 (2).

\leavevmode\hypertarget{ref-Haglich2010Detecting}{}%
Haglich, Peter, Christopher Rouff, and Laura Pullum. 2010. ``Detecting
Emergence in Social Networks.'' \emph{2010 IEEE Second International
Conference on Social Computing}. \url{https://doi.org/10/fqjq4q}.

\leavevmode\hypertarget{ref-Hamel2009Information}{}%
Hamel, Jean-Yves. 2009. ``Information and Communication Technologies and
Migration.'' \emph{Human Development Research Paper}, no. 39.
\url{https://mpra.ub.uni-muenchen.de/19175/1/MPRA_paper_19175.pdf}.

\leavevmode\hypertarget{ref-Hannides2016Voices}{}%
Hannides, Theodora, Nicola Bailey, and Dwan Kaoukji. 2016. ``Voices of
Refugees: Information and Communication Needs of Refugees in Greece and
Germany.''
\url{http://www.bbc.co.uk/mediaaction/publications-and-resources/research/reports/voices-of-refugees}.

\leavevmode\hypertarget{ref-Harney2013Precarity}{}%
Harney, Nicholas. 2013. ``Precarity, Affect and Problem Solving with
Mobile Phones by Asylum Seekers, Refugees and Migrants in Naples,
Italy.'' \emph{Journal of Refugee Studies} 26 (4).
\url{https://doi.org/10/w4q}.

\leavevmode\hypertarget{ref-Haythornthwaite2002Strong}{}%
Haythornthwaite, Caroline. 2002. ``Strong, Weak, and Latent Ties and the
Impact of New Media.'' \emph{The Information Society} 18 (5): 385--401.
\url{https://doi.org/10/bxmxht}.

\leavevmode\hypertarget{ref-HilbertQuantifying}{}%
Hilbert, Martin. n.d. ``Quantifying the Data Deluge and the Data
Drought: Background Note for the World Development Report 2016.''

\leavevmode\hypertarget{ref-Hilbert2011Worlds}{}%
Hilbert, M., and P. Lopez. 2011. ``The World's Technological Capacity to
Store, Communicate, and Compute Information.'' \emph{Science} 332
(6025). \url{https://doi.org/10/b89ttd}.

\leavevmode\hypertarget{ref-Hilhorst2010Humanitarian}{}%
Hilhorst, Dorothea, and Bram J. Jansen. 2010. ``Humanitarian Space as
Arena: A Perspective on the Everyday Politics of Aid: Humanitarian Space
as Arena.'' \emph{Development and Change} 41 (6): 1117--39.
\url{https://doi.org/10.1111/j.1467-7660.2010.01673.x}.

\leavevmode\hypertarget{ref-Hiller2004New}{}%
Hiller, Harry H., and Tara M. Franz. 2004. ``New Ties, Old Ties and Lost
Ties: The Use of the Internet in Diaspora.'' \emph{New Media \& Society}
6 (6). \url{https://doi.org/10/dszwtf}.

\leavevmode\hypertarget{ref-GSMAIntelligence2017Mobile}{}%
Intelligence, GSMA. 2017. ``The Mobile Economy: 2017.'' GSMA Mobile
Economy. GSMA.
\url{https://www.gsmaintelligence.com/research/?file=9e927fd6896724e7b26f33f61db5b9d5\&download}.

\leavevmode\hypertarget{ref-2014Inter-Agency}{}%
``Inter-Agency Rapid Assessment Report: Understanding the Information
and Communication Needs Among Idps in Northern Iraq.'' 2014. Internews.
\url{https://www.internews.org/sites/default/files/resources/Iraq_IA_CwC_Report_2014-08_web.pdf}.

\leavevmode\hypertarget{ref-Internews2011Dadaab}{}%
Internews. 2011. ``Dadaab, Kenya: Humanitarian Communications and
Information Needs Assessment Among Refugees in the Camps.''

\leavevmode\hypertarget{ref-2014Isolated}{}%
``Isolated and Misinformed, Syrian Refugees Struggle.'' 2014.
\emph{Internews}.
\url{https://www.internews.org/isolated-and-misinformed-syrian-refugees-struggle}.

\leavevmode\hypertarget{ref-Keck1998Activists}{}%
Keck, Margaret E., and Kathryn Sikkink. 1998. \emph{Activists Beyond
Borders: Advocacy Networks in International Politics}. Ithaca, N.Y:
Cornell University Press.

\leavevmode\hypertarget{ref-Keen2008Complex}{}%
Keen, David. 2008. \emph{Complex Emergencies}. Cambridge: Polity.

\leavevmode\hypertarget{ref-2009Cyberpower}{}%
Kramer, Franklin D., Stuart H. Starr, and Larry K. Wentz, eds. 2009.
\emph{Cyberpower and National Security}. Dulles, VA: Potamac Books, Inc.

\leavevmode\hypertarget{ref-Kuner2017Handbook}{}%
Kuner, Christopher, Vagelis Papakonstantinou, Lina Jasmontaite, Amy
Weatherburn, Massimo Marelli, Pierre Apraxine, Romain Bircher, et al.
2017. ``Handbook on Data Protection in Humanitarian Action.'' Geneva:
International Committee of the Red Cross.
\url{https://www.icrc.org/en/publication/handbook-data-protection-humanitarian-action}.

\leavevmode\hypertarget{ref-Lanchester2017You}{}%
Lanchester, John. 2017. ``You Are the Product.'' \emph{London Review of
Books}, August, 3--10.

\leavevmode\hypertarget{ref-Latour2005Reassembling}{}%
Latour, Bruno. 2005. \emph{Reassembling the Social: An Introduction to
Actor-Network-Theory}. Clarendon Lectures in Management Studies. Oxford
; New York: Oxford University Press.

\leavevmode\hypertarget{ref-Lewis2018Alternative}{}%
Lewis, Rebecca. 2018. ``Alternative Influence: Broadcasting the
Reactionary Right on Youtube.'' Data \& Society's Media Manipulation
Research Initiative. Data; Society.
\url{https://datasociety.net/wp-content/uploads/2018/09/DS_Alternative_Influence.pdf}.

\leavevmode\hypertarget{ref-Maitland2015Social}{}%
Maitland, Carleen, and Ying Xu. 2015. ``A Social Informatics Analysis of
Refugee Mobile Phone Use: A Case Study of Za'atari Syrian Refugee
Camp.'' \emph{43rd Research Conference on Communications, Information
and Internet Policy (TPRC)}, 10.

\leavevmode\hypertarget{ref-Massey1987social}{}%
Massey, Douglas S, and Felix García España. 1987. ``The Social Process
of International Migration.'' \emph{Science} 237 (4816).
\url{https://doi.org/10/cjpvm2}.

\leavevmode\hypertarget{ref-PeterMaurer2019Rules}{}%
Maurer, Peter, Charles Stimson, and Susan Glasser. 2019. ``Rules in War
-- a Thing of the Past?'' Center for Strategic \& International Studies.
\url{https://www.csis.org/analysis/rules-war-thing-past}.

\leavevmode\hypertarget{ref-2017Migrants}{}%
``Migrants with Mobiles: Phones Are Now Indispensable for Refugees.''
2017. \emph{The Economist}, February.

\leavevmode\hypertarget{ref-Mozur2018Genocide}{}%
Mozur, Paul. 2018. ``A Genocide Incited on Facebook, with Posts from
Myanmar's Military.'' \emph{The New York Times}, October.
\url{https://www.nytimes.com/2018/10/15/technology/myanmar-facebook-genocide.html}.

\leavevmode\hypertarget{ref-Nedelcu2012Migrants}{}%
Nedelcu, Mihaela. 2012. ``Migrants' New Transnational Habitus:
Rethinking Migration Through a Cosmopolitan Lens in the Digital Age.''
\emph{Journal of Ethnic and Migration Studies} 38 (9).
\url{https://doi.org/10/gdqfhk}.

\leavevmode\hypertarget{ref-Newell2016Information}{}%
Newell, Bryce Clayton, Ricardo Gomez, and Verónica E. Guajardo. 2016.
``Information Seeking, Technology Use, and Vulnerability Among Migrants
at the United States--Mexico Border.'' \emph{The Information Society} 32
(3). \url{https://doi.org/10/gdqfhf}.

\leavevmode\hypertarget{ref-NexonDynastic-Imperial}{}%
Nexon, Daniel. n.d. ``The Dynastic-Imperial Pathway.'' In \emph{The
Struggle for Power in Early Modern Europe: Religious Conflict, Dynastic
Empires, and International Change}.

\leavevmode\hypertarget{ref-OHCHR}{}%
``OHCHR Questions and Answers About Idps.'' n.d.
\url{https://www.ohchr.org/EN/Issues/IDPersons/Pages/Issues.aspx}.

\leavevmode\hypertarget{ref-Parker2013Humanitarian}{}%
Parker, Ben. 2013. ``Humanitarian Besieged.'' \emph{ODI HPN}, no. 59
(November): 3--5.

\leavevmode\hypertarget{ref-Pierskalla2013Technology}{}%
Pierskalla, Jan H, and Florian M Hollenbach. 2013. ``Technology and
Collective Action: The Effect of Cell Phone Coverage on Political
Violence in Africa.'' \emph{AMERICAN POLITICAL SCIENCE REVIEW} 107 (2):
207--24. \url{https://doi.org/10/f4xx7w}.

\leavevmode\hypertarget{ref-Poole2017Technology}{}%
Poole, Danielle N. 2017. ``Technology and Migration Survey.'' In
\emph{IDRG Annual Report 2017}. Den Haag: International Data
Responsibility Group Annual Conference.

\leavevmode\hypertarget{ref-Poot1996Information}{}%
Poot, Jacques. 1996. ``Information, Communication, and Networks in
International Migration System.'' \emph{The Annals of Regional Science}
30 (1). \url{https://doi.org/10/dvzgjv}.

\leavevmode\hypertarget{ref-Price2003Transnational}{}%
Price, Richard M. 2003. ``Transnational Civil Society and Advocacy in
World Politics.'' \emph{World Politics} 55 (4): 579--606.
\url{https://doi.org/10.1353/WP.2003.0024}.

\leavevmode\hypertarget{ref-Quintanilla2012Humanitarian}{}%
Quintanilla, Jacobo. 2012. ``Humanitarian Information Needs Assessment:
Zaatari Refugee.'' Internews.

\leavevmode\hypertarget{ref-Feb2018Report}{}%
``Report of the Independent International Commission of Inquiry on the
Syrian Arab Republic.'' 2018a. \emph{HRC}.

\leavevmode\hypertarget{ref-August2018Report}{}%
``Report of the Independent International Commission of Inquiry on the
Syrian Arab Republic.'' 2018b. \emph{HRC}.

\leavevmode\hypertarget{ref-Roberts2017Digital}{}%
Roberts, Hal, Brittany Seymour, Sands Alden Fish, Emily Robinson, and
Ethan Zuckerman. 2017. ``Digital Health Communication and Global Public
Influence: A Study of the Ebola Epidemic.'' \emph{Journal of Health
Communication} 22 (sup1). \url{https://doi.org/10/gdqfhh}.

\leavevmode\hypertarget{ref-Roose2017Forget}{}%
Roose, Kevin. 2017. ``Forget Washington. Facebook's Problems Abroad Are
Far More Disturbing.'' \emph{The New York Times}, October.
\url{https://www.nytimes.com/2017/10/29/business/facebook-misinformation-abroad.html}.

\leavevmode\hypertarget{ref-Ros2007Migration}{}%
Ros, Adela, Elisabet González, Antoni Marín, and Papa Sow. 2007.
``Migration and Information Flows: A New Lens for the Study of
Contemporary International Migration.'' Working Paper Series. Barcelona:
Internet Interdisiplinary Institute.
\url{http://www.uoc.edu/in3/dt/eng/ros_gonzalez_marin_sow.pdf}.

\leavevmode\hypertarget{ref-Sandvik2016Humanitarian}{}%
Sandvik, Kristin Bergtora. 2016. ``The Humanitarian Cyberspace:
Shrinking Space or Expanding Frontier?'' \emph{Third World Quarterly} 37
(1). \url{https://doi.org/10/gdqfhg}.

\leavevmode\hypertarget{ref-Sandvik2017Do}{}%
Sandvik, Kristin Bergtora, Katja Lindskov Jacobsen, and Sean Martin
McDonald. 2017. ``Do No Harm: A Taxonomy of the Challenges of
Humanitarian Experimentation.'' \emph{International Review of the Red
Cross}, October, 1--26. \url{https://doi.org/10/gddxwv}.

\leavevmode\hypertarget{ref-Schapendonk2007Migration}{}%
Schapendonk, Joris, and David van Moppes. 2007. ``Migration and
Information: Images of Europe, Migration Encouraging Factors and En
Route Information Sharing.'' Working Papers Migration and Development
Series. Nijmegen: Radboud University.

\leavevmode\hypertarget{ref-Schmitt2016Community-Level}{}%
Schmitt, Paul, Daniel Iland, Elizabeth Belding, Brian Tomaszewski, Ying
Xu, and Carleen Maitland. 2016. ``Community-Level Access Divides: A
Refugee Camp Case Study.'' In \emph{Proceedings of the Eighth
International Conference on Information and Communication Technologies
and Development}. Ann Arbor, MI: ICTD '16 Eighth International
Conference on Information; Communication Technologies; Development
Conference; ACM Press. \url{https://doi.org/10.1145/2909609.2909668}.

\leavevmode\hypertarget{ref-Shapiro2015Is}{}%
Shapiro, Jacob N., and Nils B. Weidmann. 2015. ``Is the Phone Mightier
Than the Sword? Cellphones and Insurgent Violence in Iraq.''
\emph{INTERNATIONAL ORGANIZATION} 69 (2): 247--74.
\url{https://doi.org/10/f7bnnr}.

\leavevmode\hypertarget{ref-Slim2003Is}{}%
Slim, Hugo. 2003. ``Is Humanitarianism Being Politicised? A Reply to
David Rieff.'' \emph{The Dutch Red Cross Symposium on Ethics in Aid},
no. October: 1--8.

\leavevmode\hypertarget{ref-Solon2017How}{}%
Solon, Olivia. 2017. ``How Syria's White Helmets Became Victims of an
Online Propaganda Machine.'' \emph{The Guardian}, December.
\url{http://www.theguardian.com/world/2017/dec/18/syria-white-helmets-conspiracy-theories}.

\leavevmode\hypertarget{ref-Specia2017War}{}%
Specia, Megan, and Paul Mozur. 2017. ``A War of Words Puts Facebook at
the Center of Myanmar's Rohingya Crisis.'' \emph{The New York Times},
October.
\url{https://www.nytimes.com/2017/10/27/world/asia/myanmar-government-facebook-rohingya.html}.

\leavevmode\hypertarget{ref-Starbird2017Examining}{}%
Starbird, Kate. 2017. ``Examining the Alternative Media Ecosystem
Through the Production of Alternative Narratives of Mass Shooting Events
on Twitter.'' In \emph{11th International AAAI Conference on Web and
Social Media}. Montreal: 11th International AAAI Conference on Web;
Social Media.
\url{http://faculty.washington.edu/kstarbi/Alt_Narratives_ICWSM17-CameraReady.pdf}.

\leavevmode\hypertarget{ref-KateStarbird2019Disinformation}{}%
Starbird, Kate, Ahmer Arif, and Tom Wilson. 2019. ``Disinformation as
Collaborative Work: Surfacing the Participatory Nature of Strategic
Information Operations.'' In. Vol. CSCW. PACMHCI.
\url{https://faculty.washington.edu/kstarbi/StarbirdArifWilson_DisinformationasCollaborativeWork-CameraReady-Preprint.pdf}.

\leavevmode\hypertarget{ref-Starbird2018Ecosystem}{}%
Starbird, Kate, Ahmer Arif, Tom Wilson, Katherine Van Koevering, Katya
Yefimova, and Daniel P. Scarnecchia. 2018. ``Ecosystem or Echo-System?
Exploring Content Sharing Across Alternative Media Domains.'' In
\emph{12th International AAAI Conference on Web and Social Media
(ICWSM-18)}. Stanford, CA: 12th International AAAI Conference on Web;
Social Media (ICWSM-18); Association for the Advancement of Artificial
Intelligence Publications.
\url{https://www.aaai.org/ocs/index.php/ICWSM/ICWSM18/paper/view/17836}.

\leavevmode\hypertarget{ref-Sunstein2009Conspiracy}{}%
Sunstein, Cass R., and Adrian Vermeule. 2009. ``Conspiracy Theories:
Causes and Cures*.'' \emph{Journal of Political Philosophy} 17 (2):
202--27. \url{https://doi.org/10/bdd3hg}.

\leavevmode\hypertarget{ref-Thulin2014Virtual}{}%
Thulin, Eva, and Bertil Vilhelmson. 2014. ``Virtual Practices and
Migration Plans: A Qualitative Study of Urban Young Adults.''
\emph{Population, Space and Place} 20 (5): 389--401.
\url{https://doi.org/10/f24m9s}.

\leavevmode\hypertarget{ref-Tilly1991Transplanted}{}%
Tilly, Charles. 1991. ``Transplanted Networks.'' In \emph{Immigration
Reconsidered: History, Sociology, and Politics}, edited by Virginia
Yans-McLaughlin. New York: Oxford University Press.
\url{http://www.oxfordscholarship.com/view/10.1093/acprof:oso/9780195055108.001.0001/acprof-9780195055108-chapter-4}.

\leavevmode\hypertarget{ref-UNHCR2015Syrian}{}%
UNHCR. 2015. ``Syrian Refugee Arrivals in Greece - Preliminary
Questionnaire Findings April-September 2015.'' UNHCR.
\url{https://data2.unhcr.org/en/documents/details/46542}.

\leavevmode\hypertarget{ref-Vernon2016Connecting}{}%
Vernon, Alan, Kamel Deriche, and Samantha Eisenhauer. 2016. ``Connecting
Refugees. How Internet and Mobile Connectivity Can Improve Refugee
Well-Being and Transform Humanitarian Action.'' Geneva: UNHCR.
\url{https://www.unhcr.org/5770d43c4.pdf}.

\leavevmode\hypertarget{ref-Vinck2018Engaging}{}%
Vinck, Patrick, Anne Bennett, and Jacobo Quintanilla. 2018. ``Engaging
with People Affected by Armed Conflicts and Other Situations of
Violence: Recommendations for Humanitarian Origanizations and Donors in
the Digital Era.'' Geneva: International Committee of the Red Cross,
Harvard Humanitarian Initiative.
\url{https://www.icrc.org/en/download/file/69676/engaging-with-people-in-armed-conflict-recommendationt.pdf}.

\leavevmode\hypertarget{ref-Wall2017Syrian}{}%
Wall, Melissa, Madeline Otis Campbell, Dana Janbek, Madeline Otis
Campbell, and Dana Janbek. 2017. ``Syrian Refugees and Information
Precarity.'' \emph{New Media \& Society} 19 (2).
\url{https://doi.org/10/gc92db}.

\leavevmode\hypertarget{ref-Weedon2017Information}{}%
Weedon, Jen, William Nuland, and Alex Stamos. 2017. ``Information
Operations and Facebook.'' Facebook.
\url{https://fbnewsroomus.files.wordpress.com/2017/04/facebook-and-information-operations-v1.pdf?utm_source=Daily+Lab+email+list\&utm_campaign=c7a8276ae4-dailylabemail3\&utm_medium=email\&utm_term=0_d68264fd5e-c7a8276ae4-395936553}.

\leavevmode\hypertarget{ref-WelchChris2017Facebook}{}%
Welch, Chris. 2017. ``Facebook Crosses 2 Billion Monthly Users - the
Verge.'' \emph{The Verge}.
\url{https://www.theverge.com/2017/6/27/15880494/facebook-2-billion-monthly-users-announced}.

\leavevmode\hypertarget{ref-WilsonAssembling}{}%
Wilson, Tom, Kaitlyn Zhou, and Kate Starbird. n.d. ``Assembling
Strategic Narratives: Information Operations as Collaborative Work
Within an Online Community'' 2: 25.

\leavevmode\hypertarget{ref-Winner1980Do}{}%
Winner, Langdon. 1980. ``Do Artifacts Have Politics?'' \emph{Daedalus}
109 (1): 148--64.

\leavevmode\hypertarget{ref-Wissink2017In}{}%
Wissink, Marieke, and Valentina Mazzucato. 2017. ``In Transit: Changing
Social Networks of Sub-Saharan African Migrants in Turkey and Greece.''
\emph{Social Networks}, March. \url{https://doi.org/10/gc8kt9}.

\leavevmode\hypertarget{ref-Zuckerberg2017Mark}{}%
Zuckerberg, Mark. 2017. ``Mark Zuckerberg -as of This Morning, the
Facebook Community Is...'' \emph{Facebook}.
\url{https://www.facebook.com/zuck/posts/10103831654565331}.
\end{cslreferences}

\end{document}
